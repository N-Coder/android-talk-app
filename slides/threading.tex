\section{Threading}
\frame{
	\frametitle{Threading}
	TODO
}